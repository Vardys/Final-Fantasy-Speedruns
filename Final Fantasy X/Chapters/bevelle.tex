\chapter{Bevelle}\label{ch:bevelle}

\begin{enumerate}
	\item Don't skip FMV
	\item Use a Mega-Potion
	\item \textit{With \sleepingpowder{}:}
\end{enumerate}
\begin{battle}{Guard Fights - \sleepingpowder}
	\begin{itemize}
		\item \textit{Fights 1 and 3:}
		      \begin{itemize}
			      \tidusf Attack
			      \item Defend or use Distillers
		      \end{itemize}
		\item \textit{Fights 2 and 4:}
		      \begin{itemize}
			      \tidusf Attack
			      \rikkuf \sleepingpowder
			      \kimahrif \bombcore\ /\silencegrenade\ /\smokebomb\ / Distiller\
		      \end{itemize}
		\item \textit{Fight 5:}
		      \begin{itemize}
			      \tidusf Haste \rikku
			      \rikkuf Throw Items x2
			      \tidusf Attack
		      \end{itemize}
	\end{itemize}
\end{battle}
\bothvfill
\lossvfill
\winvfill
\ 
\bothcb
\wincb
\losscb
\ 
\ \bothnewline \winnewline \lossnewline
\begin{enumerate}[resume]
	\item \textit{Without \sleepingpowder{}:}
	      \begin{itemize}
		      \item \formation{\tidus}{\rikku}{\auron} \textit{unless \lulu\ doesn't have at least 35 levels, then } \formation{\tidus}{\rikku}{\lulu}
	      \end{itemize}
\end{enumerate}
\begin{battle}{Guard Fights - No \sleepingpowder}
	\begin{itemize}
		\item \textit{Fights 1 and 3:}
		      \begin{itemize}
			      \tidusf Attack
			      \item Defend or use Distillers
		      \end{itemize}
		\item \textit{Fights 2 and 4:}
		      \begin{itemize}
			      \switch{\tidus}{\kimahri}
			      \kimahrif \bombcore\ /\silencegrenade\ /\smokebomb\
			      \switch{\rikku}{\tidus}
			      \tidusf Attack
			      \kimahrif Repeat
			      \item If Underdamaged anyone, use another Throwable
		      \end{itemize}
		\item After the second fight, \formation{\tidus}{\rikku}{\lulu}
		\item \textit{Fight 5:}
		      \begin{itemize}
			      \switch{\tidus}{\rikku}
			      \rikkuf \bombcore\ /\silencegrenade\ /\smokebomb\ x2
			      \switch{\kimahri}{\tidus}
			      \tidusf Attack
		      \end{itemize}
	\end{itemize}
\end{battle}
\begin{enumerate}[resume]
	\item \sd, \skippablefmv, \sd\ on \yuna\ dialogue. \skippablefmv, \sd. Use lift, \sd.
\end{enumerate}
\bothvfill
\winvfill
\lossvfill
\ 
\begin{trial}
	\begin{itemize}
		\item For all of these you can Hold X instead of pressing it when you get onto the directional pad. The counts of which junction are 1-indexed, starting at the beginning of each area.
		\item Push the pedestal in
		\item Press X
		\item Go left at the second junction
		\item Take sphere, push pedestal back into the junction
		\item At the third junction, go back
		\item Go left at the second junction
		\item Place sphere into wall, push pedestal back
		\item Go left at the first junction
		\item Go left (there are two cycles; the arrow on the third junction needs to point to the right when you press x)
		\item At the third junction and go right
		\item Take glyph sphere from wall, push pedestal back onto the road
		\item At the fourth junction go right
		\item Place glyph sphere into pedestal
		\item Take Bevelle sphere from pedestal
		\item Place Bevelle sphere into the wall
		\item Take the glyph sphere
		\item Place into the next wall
		\item Take Destruction sphere from the new wall %where to put it
		\item Take Bevelle sphere from old wall
		\item Push pedestal back and fall off the edge
		\item Go straight
		\item At the third junction go right
		\item Place destruction sphere into wall
		\item Push pedestal back and fall off the edge
		\item Go straight
		\item At the second junction go right
		\item Push pedestal
		\item Go up the stairs, open the chest
	\end{itemize}
\end{trial}
\begin{enumerate}[resume]
	\item \sd, name \bahamut, don't save, \sd
\end{enumerate}
\lossvfill
\ 
\losscb
\ \lossnewline \ 